\def \caixa {\nopagebreak\hfill\textcolor{lightgray}{$\Box$}}


\documentclass[a6paper]{article}
\usepackage[17pt]{extsizes}
\usepackage[paperwidth=136.6mm,paperheight=76.4mm,left=0.18cm,top=.1cm,right=0.3cm,bottom=0.15cm,marginparwidth=0mm,marginparsep=0mm]{geometry}

 	\pagestyle{empty}
    \makeatletter
    \newif\if@sectionused \@sectionusedfalse
    \newif\if@subsectionused \@subsectionusedfalse
    \newif\if@subsubsectionused \@subsubsectionusedfalse
    \newif\if@paragraphused \@paragraphusedfalse
    \let\oldsection\section
    \let\oldsubsection\subsection
    \let\oldsubsubsection\subsubsection
    \let\oldparagraph\paragraph
    \renewcommand{\section}{
    \if@sectionused\clearpage\fi
    \@sectionusedtrue\@subsectionusedfalse\@subsubsectionusedfalse\@paragraphusedfalse
    \def \fimsec {\hfill\textcolor{lightgray}{$\Box$}}
    \def \fimsubsec {}
    \def \fimsubsubsec {}
    \ifdefined\zeraRodape\setcounter{footnote}{0}\fi
    \oldsection}
    \renewcommand{\subsection}{
    \if@subsectionused\clearpage\fi
    \@subsectionusedtrue\@subsubsectionusedfalse\@paragraphusedfalse
    \def \fimsubsec {\hfill\textcolor{lightgray}{$\Box$}}
    \def \fimsubsubsec {}
    \oldsubsection}
    \renewcommand{\subsubsection}{
    \if@subsubsectionused\clearpage\fi
    \@subsubsectionusedtrue\@paragraphusedfalse
    \def \fimsubsubsec {\hfill\textcolor{lightgray}{$\Box$}}
    \oldsubsubsection}
    %\renewcommand{\paragraph}{\if@paragraphused\clearpage\fi\@paragraphusedtrue\oldparagraph}
    \makeatother


\pdfminorversion=6
\pdfcompresslevel=9
\pdfobjcompresslevel=3

\usepackage[utf8]{inputenc}
\usepackage{libertine}
\usepackage{libertinust1math}
\usepackage[T1]{fontenc}

% para o markdown
\usepackage[footnotes,definitionLists,hashEnumerators,smartEllipses, hybrid]{markdown}
  \markdownSetup{rendererPrototypes={
    link = {\href{#2}{#1}}
  }}
\newcommand{\hash}{\#}

\usepackage[brazil]{babel}
\usepackage[babel=true,kerning=true]{microtype}

\usepackage{tocloft} % para mudar o titulo do indice
\usepackage{etoc} % para saber se tem indice ou não
% https://tex.stackexchange.com/questions/94961/how-to-check-in-latex-whether-the-table-of-content-is-empty-or-not-before-added
\etocchecksemptiness % do not display empty local table of contents
\etocnotocifnotoc % do not display empty global table of contents

% usado para a capa...
\usepackage{tikz}
\usetikzlibrary{positioning}
\usepackage{varwidth}
\pgfdeclarelayer{bg}    % declare background layer
\pgfdeclarelayer{front} 
\pgfsetlayers{bg,main,front}  % set the order of the layers (main is the standard layer)
\usepackage[hidelinks,
pdftex,
  pdfauthor={\mautor},
  pdftitle={\mtitulo},
  pdfsubject={Ancap/acao-humana},
  pdfkeywords={},
  pdfproducer={source in markdown},
  pdfcreator={AncapChannel}
]{hyperref}
%\usepackage{calc}
\usetikzlibrary{calc}
\usepackage{graphicx}
\usepackage{calc}
\usepackage{ifthen}


\DeclareUrlCommand\Hurl{%
  \def\UrlLeft{\langle}%
  \def\UrlRight{\rangle}%
}
\renewcommand*{\UrlFont}{\ttfamily\scriptsize\relax}

%\usepackage{psvectorian}
%\let\clipbox\relax
\usepackage{pgfornament}


% usado par ao índice
\addto\captionsbrazil{
  \renewcommand{\contentsname}
    {}% A frase que aparece no lugar de "Conteúdo" no índice
}

% usado para limitar o tamanho das imagens
\usepackage[export]{adjustbox}

% usado para esticar e cortar imagens
% https://tex.stackexchange.com/questions/60918/how-to-scale-and-then-trim-an-image
\newlength{\oH}
\newlength{\oW}
\newlength{\rH}
\newlength{\rW}
\newlength{\cH}
\newlength{\cW}
\newcommand\ClipImage[3]{% width, height, image
\settototalheight{\oH}{\includegraphics{#3}}%
\settowidth{\oW}{\includegraphics{#3}}%
\setlength{\rH}{\oH * \ratio{#1}{\oW}}%
\setlength{\rW}{\oW * \ratio{#2}{\oH}}%
\ifthenelse{\lengthtest{\rH < #2}}{%
    \setlength{\cW}{(\rW-#1)*\ratio{\oH}{#2}}%
    \adjincludegraphics[height=#2,clip,trim=0 0 \cW{} 0]{#3}%
}{%
    \setlength{\cH}{(\rH-#2)*\ratio{\oW}{#1}}%
    \adjincludegraphics[width=#1,clip,trim=0 \cH{} 0 0]{#3}%
}%
}


%\def \titulo {Título do Artigo} 
%\def \autor {} 
%\def \tradutor {} 
%\def \url {} 
%\def \CriadorDestePDF {}

% usado para fonte decorativa
\usepackage{lettrine}
%\usepackage{GoudyIn}
\usepackage{Zallman}
%\renewcommand{\LettrineFontHook}{\GoudyInfamily{}}
\renewcommand{\LettrineFontHook}{\Zallmanfamily{}}

\newcommand{\DECORAR}[3][]{\lettrine[lines=3,loversize=.115,#1]{#2}{#3}}

% para listar o link da discussão e a data:
\makeatletter
\def\blfootnote{\xdef\@thefnmark{$ \sim $}\@footnotetext}
\makeatother
% https://tex.stackexchange.com/questions/250221/supressing-the-footnote-number

\makeatletter
\newcommand*{\myfnsymbolsingle}[1]{%
  \ensuremath{%
    \ifcase#1% 0
    \or % 1
      *%   
    \or % 2
      \dagger
    \or % 3  
      \ddagger
    \or % 4   
      \mathsection
    \or % 5
      \mathparagraph
    \else % >= 6
      \@ctrerr  
    \fi
  }%   
}   
\makeatother
\newcommand*{\myfnsymbol}[1]{%
  \myfnsymbolsingle{\value{#1}}%
}
\makeatletter
\newcommand*{\greekfnsymbolsingle}[1]{%
  \ensuremath{%
  \ifcase#1\or\alpha\or\beta\or\gamma\or\delta\or\varepsilon
    \or\zeta\or\eta\or\theta\or\iota\or\kappa\or\lambda
    \or\mu\or\nu\or\xi\or o\or\pi\or\varrho\or\sigma
    \or\tau\or\upsilon\or\phi\or\chi\or\psi\or\omega
    \else\@ctrerr\fi
  }%   
}   
\makeatother
\newcommand*{\greekfnsymbol}[1]{%
  \greekfnsymbolsingle{\value{#1}}%
}

\usepackage[hang,flushmargin,multiple]{footmisc}
\newcommand{\nota}[1]{[#1]}
\let\oldfootnote\footnote
\renewcommand{\footnote}[1]{\oldfootnote{#1\caixa}}


% cor
\usepackage[prefix=s]{xcolor-solarized}
\usepackage{pagecolor}
\ifdefined\light
  \colorlet{srebase03}{sbase3}
  \colorlet{srebase02}{sbase2}
  \colorlet{srebase01}{sbase1}
  \colorlet{srebase00}{sbase0}
  \colorlet{srebase0}{sbase00}
  \colorlet{srebase1}{sbase01}
  \colorlet{srebase2}{sbase02}
  \colorlet{srebase3}{sbase03}
\else
  \colorlet{srebase03}{sbase03}
  \colorlet{srebase02}{sbase02}
  \colorlet{srebase01}{sbase01}
  \colorlet{srebase00}{sbase00}
  \colorlet{srebase0}{sbase0}
  \colorlet{srebase1}{sbase1}
  \colorlet{srebase2}{sbase2}
  \colorlet{srebase3}{sbase3}
\fi

